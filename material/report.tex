\documentclass[12pt,letterpaper]{article}
\usepackage[UTF8]{ctex}
\usepackage{fullpage}
\usepackage[top=2cm, bottom=4.5cm, left=2.5cm, right=2.5cm]{geometry}
\usepackage{amsmath,amsthm,amsfonts,amssymb,amscd}
\usepackage{lastpage}
\usepackage{enumerate}
\usepackage{fancyhdr}
\usepackage{mathrsfs}
\usepackage{xcolor}
\usepackage{graphicx}
\usepackage{listings}
\usepackage{minted}
\usepackage{hyperref}
\usepackage{dot2texi}
\usepackage{tikz}
\usepackage{float}
\usepackage[pdf]{graphviz}
\usetikzlibrary{automata,shapes,arrows}

\hypersetup{%
  colorlinks=true,
  linkcolor=blue,
  linkbordercolor={0 0 1}
}

\renewcommand\lstlistingname{Algorithm}
\renewcommand\lstlistlistingname{Algorithms}
\def\lstlistingautorefname{Alg.}

\lstdefinestyle{Matlab}{
    language        = matlab,
    frame           = lines,
    basicstyle      = \footnotesize,
    keywordstyle    = \color{blue},
    stringstyle     = \color{green},
    commentstyle    = \color{red}\ttfamily
}
\setlength{\parindent}{0.0in}
\setlength{\parskip}{0.05in}

% Edit these as appropriate
\newcommand\course{自然语言处理基础}
\newcommand\hwnumber{1}                  % <-- homework number
\newcommand\name{张子沐}                 % <-- Name
\newcommand\ID{2200013115}           % <-- ID

\pagestyle{fancyplain}
\headheight 35pt
\lhead{\name\\\ID}
\chead{\textbf{\Large Homework \hwnumber}}
\rhead{\course \\ \today}
\lfoot{}
\cfoot{}
\rfoot{\small\thepage}
\headsep 1.5em





\begin{document}

\section{数据预处理}

对于语料库中的文本,首先进行如下的预处理操作。

\begin{enumerate}[a).]
    \item 取出文本中的所有标点符号,并用空格代替。
    \item 用自然语言处理库 nltk 对文本进行切词。
    \item 将所有单词的字母全部转变为小写字母。
    \item 如果某个单词不是由字母组成,则将其过滤。
    \item 将单词的形式进行标准化。
\end{enumerate}

\section{特征提取}

采用 sklearn 中的 \verb|CountVectorizer| 和 \verb|TfidfTransformer| 对经过预处理过后的语料库进行特征提取,取语料库的出现频率最高的 20000 个 unigram 作为特征词,并将这些特征词的 Tf-IDF 作为文本的特征。

值得说明的是,sklearn 中的 \verb|CountVectorizer| 会对语料库中的停用词进行过滤,所以在数据预处理中没有考虑停用词。

\section{LogLinear 模型的实现}

\subsection{模型参数}

模型的参数是一个大小为 $4 \times d$ 的 Numpy 矩阵 $\boldsymbol{P}$,其中 $d=20000$ 表示特征的维度。

\subsection{前向传播}

模型的输入:大小为 $b\times d$ 的矩阵 $\boldsymbol{D}$,其中 $b$ 表示一个 batch 所包含的数据条数。

模型的输出:
\begin{equation*}
    \mathrm{LogLinear(\boldsymbol{D})} = \mathrm{SoftMax}(\boldsymbol{D}\times \boldsymbol{P}^T)
\end{equation*}

\subsection{反向传播}

设矩阵
\begin{equation*}
    \boldsymbol{P} = \begin{bmatrix}
    \lambda_1\\ \lambda_2 \\ \lambda_3 \\ \lambda_4
    \end{bmatrix}
\end{equation*}

其中 $\lambda_i$ 表示矩阵 $\boldsymbol{P}$ 的第 $i$ 个行向量,对应着标签为第 $i$ 类的特征参数。同时,设矩阵 $\boldsymbol{D}$ 的第 $i$ 个行向量为 $f_i$,表示第 $i$ 条数据,其真实的特征为 $y_i$。

则优化目标为
\begin{equation*}
    \min\limits_{\lambda_{1\sim 4}} \Phi = \sum_{i=1}^b \left(-f_i \cdot \lambda_{y_i} + \log\left(\sum_{j=1}^4 e^{f_i \cdot \lambda_j}\right) \right)
\end{equation*}

求梯度:
\begin{equation*}
    \frac{\partial \Phi}{\partial \lambda_j} = \sum_{i=1}^n \left( -f_i\cdot \delta_{[y_i=j]} + f_i \frac{e^{f_i\cdot \lambda_j}}{\sum\limits_{k=1}^4 e^{f_i\cdot \lambda_k}} \right)
\end{equation*}

在每次迭代中进行一次梯度下降:
\begin{equation*}
    \lambda_i \gets \lambda_i - \frac{\partial \Phi}{\partial \lambda_i}\times \mathrm{lr}
\end{equation*}

其中,$\mathrm{lr}$ 表示学习率。


\begin{minted}{python}
class Log_Linear:
    def __init__(self):
        self.params = np.random.rand(4, feature_size).astype(np.float64)
    def forward(self, data_batch : np.ndarray, labels_batch : np.ndarray, require_cache=True):
        self.product = data_batch @ np.transpose(self.params)
        result = np.argmax(self.product, axis=1)
        acc = float(np.sum(result == labels_batch)) / labels_batch.shape[0]
        if require_cache == False:
            return result, acc
        self.cache = data_batch
        self.labels_cache = labels_batch
        self.sum_exp = np.sum(np.exp(self.product), axis=1)
        acc_params = self.params[labels_batch]
        loss = float(np.mean(-np.sum(data_batch * acc_params, axis=1) + np.log(self.sum_exp)))
        return result, loss, acc
    def backward(self):
        self.prob = np.exp(self.product) / np.expand_dims(self.sum_exp, axis=1)
        self.grad = np.zeros_like(self.params, dtype=np.float64)
        for i in range(self.cache.shape[0]):
            self.grad += np.outer(self.prob[i], self.cache[i])
            self.grad[self.labels_cache[i]] -= self.cache[i]

        self.params -= self.grad * lr
\end{minted}

\subsection{训练策略}

采用随机梯度下降法对模型进行训练。在一次 epoch 中,首先随机打乱训练集,每次取出 1024 条数据喂给模型进行训练,直至遍历完整个训练集。一共进行 4 次 epoch,随着 epoch 的增加,逐渐降低学习率。

数据迭代器实现如下:

\begin{minted}{python}
class Data_Iterator:
    def __init__(self, data, labels, batch_size):
        self.data = data
        self.labels = labels
        self.btch_siz = batch_size
    def __iter__(self):
        self.order = np.random.permutation(data_size)
        self.idx = 0
        return self
    def __len__(self):
        return int(np.ceil(data_size / self.btch_siz))
    def __next__(self):
        if self.idx == data_size:
            raise StopIteration
        nxt = min(self.idx + self.btch_siz, data_size)
        slce = self.order[self.idx : nxt]
        self.idx = nxt
        return self.data[slce].astype(np.float64), self.labels[slce]
\end{minted}

\section{实验结果}

一共进行 4 次实验,实验结果如下。

\begin{table}[h!]
    \centering
    \begin{tabular}{c|c}
        \textbf{次数} & \textbf{最后一次迭代的正确率}\\
        \hline
        1 & 0.896\\
        2 & 0.873\\
        3 & 0.880\\
        4 & 0.927\\
        统计 & 0.894$\pm$0.0208\\
    \end{tabular}
    \caption{在训练集上的表现}
    \label{tab:my_label}
\end{table}


\begin{table}[h!]
    \centering
    \begin{tabular}{c|c|c}
        \textbf{次数} & \textbf{正确率} & \textbf{四个标签的平均 F1 分数}\\
        \hline
        1 & 0.896 & 0.896\\
        2 & 0.896 & 0.896\\
        3 & 0.897 & 0.896\\
        4 & 0.896 & 0.896\\
        统计 & 0.896$\pm$0.0004 &0.896$\pm$0 \\
    \end{tabular}
    \caption{在测试集上的表现}
    \label{tab:my_label}
\end{table}

\end{document}
